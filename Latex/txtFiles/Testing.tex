\section{Testing}

\subsection{Delivery Queue}

Nachrichten in der Delivery Queue enthalten nur zwei Timestamps 

initDLQ/2,expectedNr/1,push2DLQ/3, deliverMSG/4, listDLQ/1, lengthDLQ/1, delDLQ/1

1. initDLQ
-> Prüfen, ob Prozess gestartet wurde indem geprüft wird ob dieser beendet werden kann

2. expectedNr
-> Liste erstellen mit mindestens einem Element N und prüfen, ob expNr N+1 ist
-> leere Liste erstellen und prüfen, ob expNr = 1 ist

3. push2DLQ
-> Liste erstellen und prüfen, ob Element danach in Liste enthalten ist (auf richtige Position und Struktur achten)
-> leere Liste erstellen und prüfen, ob Element eingefügt wurde und list.size = 1 ist 
-> MaxSize und ActSize prüfen
-> Prüfen ob erstes Element bei erreichen von MaxSize gelöscht wird 

4. deliverMSG
-> Wenn Nachricht nicht vorhanden, dann nächstgrößere nehmen
-> was passiert bei leerer Liste 
-> Stimmt die an den Clienten gesendete Nachrichtennummer mit der zurückgegebenen überein
-> Prüfen ob Nachricht Timestamps enthaehlt (einer mehr als beim eingehen)
-> MaxSize und ActSize prüfen

5. listDLQ
-> leere Liste prüfen 
-> Rückgabewert mit einer erstellten Liste vergleichen 

6. lengthDLQ
-> leere Liste 
-> mit erlang Funktion vergleichen 

7. delDLQ
-> Prüfen ob dlqPid nach Aufruf noch erreichbar ist 

Reihenfolge:

initDLQ
expectedNr -> leere Liste 
listDLQ -> leere Liste
lengthDLQ -> leere Liste
push2DLQ -> einige Elemente auffüllen und testen 
listDLQ 
lengthDLQ
deliverMSG -> verdrehte Reihenfolge und mehr als vorhanden
delDLQ

\subsection{Holdback Queue}
